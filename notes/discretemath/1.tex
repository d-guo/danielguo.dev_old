\documentclass[12pt]{article}[letterpaper]

\addtolength{\topmargin}{-12mm}
\addtolength{\oddsidemargin}{-12mm}
\addtolength{\evensidemargin}{-12mm}
\addtolength{\textwidth}{24mm}
\addtolength{\textheight}{12mm}

\usepackage{amsfonts}
\usepackage{amsmath}
\usepackage{amssymb}
\usepackage[thmmarks, amsmath, thref]{ntheorem}
\usepackage{mathtools}
\usepackage{enumerate}
\usepackage{multicol}
\usepackage{xcolor, framed}
\usepackage{setspace}
\linespread{1}

\definecolor{shadecolor}{RGB}{225, 225, 225}

\newcommand{\sect}[1]{\noindent \textbf{Section #1}}
\newcommand{\ex}[1]{\item[\textbf{#1}]}
\newcommand{\corr}[1]{\indent \textbf{\textit{correction #1}}}
\newcommand{\set}[1]{\{#1\}}
\newcommand{\spann}[1]{\text{span}(#1)}
\newcommand{\dimm}[1]{\text{dim}(#1)}
\newcommand{\nulll}[1]{\text{null}(#1)}
\newcommand{\rangee}[1]{\text{range}(#1)}
\newcommand{\degg}[1]{\text{deg}(#1)}
\newcommand{\prob}[1]{\mathbb{P}(#1)}
\newcommand{\expec}[1]{\mathbb{E}[#1]}
\newcommand{\var}[1]{\text{Var}(#1)}
\newcommand{\tvs}{\textvisiblespace}
\newcommand{\comp}{\mathsf{C}}
\newcommand{\perm}[1]{\text{Perm}(#1)}
\newcommand{\Alpha}{\mathfrak{A}}
\newcommand{\itv}{\mathit{v}}

\newtheorem{theorem}{Theorem}
\theoremstyle{definition}
\newtheorem{definition}{Definition}
\newtheorem{example}{Example}
\theoremstyle{remark}
\newtheorem{remark}{Remark}
\theoremstyle{exercise}
\newtheorem{exercise}{Exercise}[section]

\theoremstyle{nonumberplain}
\theoremheaderfont{\itshape}
\theorembodyfont{\upshape}
\newtheorem{solution}{Solution}

\DeclarePairedDelimiter{\ceil}{\lceil}{\rceil}
\DeclarePairedDelimiter{\floor}{\lfloor}{\rfloor}

\begin{document}
\begin{center}
{\Large \textbf{An Introduction to Proofs and Proof Techniques}} \\[5mm]
Daniel Guo \\[4mm]
December 21, 2019
\end{center}

\noindent This segment introduces the concept of a proof and the basic proof techniques that are used in discrete math. I include (weak) mathematical induction here because it is a useful technique which I'd like to be able to use in the coming segments. In the future, there will be a segment going into much more detail on induction and well ordering.

\section{What is a Proof?}
I don't know about a formal definition of a proof, but basically it's an argument which guarantees the truth of a proposition according to certain rules.
\begin{definition}
A \textit{proposition} is a statement which can be true or false.
\end{definition}
\begin{example}
dsf
\end{example}
\begin{solution}
trivial
\end{solution}

\end{document}